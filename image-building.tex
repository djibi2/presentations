\section{Image Building}

Servers launched with Heat can use any disk image, so the building of disk images is orthogonal to what Heat does. Building custom images allows users to avoid time-consuming installation processes at boot, but not all clouds allow custom images. There will always be demand for customising images at runtime purely for convenience as well, especially during testing.

At the begining of the Heat project, we typically used Oz TDL files to build disk images containing the \texttt{heat-cfntools}. Those tools have since been packaged, and can now be obtained from RPMs or from PyPI (meaning a custom image is not strictly necessary).

Around a year ago, Tomas Sedovic built a tool\footnote{The tool, known as \texttt{heat-prebuild}, is available \href{https://github.com/sdake/heat-prebuild}{on GitHub}.} for converting the \texttt{heat-cfntools} configuration data in Heat templates into a TDL format, building a disk image with Oz and uploading the result to Glance\footnote{The OpenStack Image Store service.}. However this tool is now unmaintained for want of interest.

The Heat project is moving to align more with the rest of OpenStack, by using the \texttt{diskimage-builder} tool developed as part of TripleO. However, there is no equivalent of Oz's TDL format. \texttt{diskimage-builder} builds images in a chroot jail by running a user-configured set of scripts at each stage of the process.
