\section{Matahari Roadmap}

A detailed list of what tasks are currently being worked on can be found on the project \href{https://github.com/matahari/matahari/wiki/Backlog}{backlog}.  There are some additional things worth mentioning, though.

The Matahari team plans to get more involved in the development of QMF itself.  We have written some code that helps make writing QMF (and D-Bus) agents easier, and we would like to get some of those enhancements merged back into QMF.  Some of the other things we would like to do in QMF are to help drive adoption.  This includes splitting QMF out from the Qpid project, a re-work in C (it is currently C++), and potentially supporting other underlying AMQP implementations to be able to interoperate with AMQP infrastructure that doesn't use the same version of AMQP as Qpid.

In addition to working on the infrastructure, we will also be continuing to write agents.  Matahari will grow as a collection of general-purpose cross-platform agents and we will be working with with other projects to expose their APIs via the Matahari infrastructure.

\subsection{More Information}

Here is a list of other places you can go to get more information and participate in the project:

\begin{itemize}
\item The project home page is \url{http://matahariproject.org/}.
\item You can find the code on \href{https://github.com/matahari/matahari}{GitHub}.
\item You can follow patches and project-related discussion on the \href{https://fedorahosted.org/mailman/listinfo/matahari}{Matahari mailing list}. 
\item You can join the \texttt{\#matahari} IRC channel on OFTC.
\end{itemize}
