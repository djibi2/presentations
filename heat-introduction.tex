\documentclass{tufte-handout}

\usepackage{fancyvrb}
\fvset{fontsize=\normalsize}

\usepackage{hyperref}
\hypersetup{colorlinks=true}

\usepackage{graphicx}

% Hide page numbers in ToC
\let\Contentsline\contentsline
\renewcommand\contentsline[3]{\Contentsline{#1}{#2}{}}

\title{OpenStack Orchestration with Heat}
\author{\href{mailto:zbitter@redhat.com}{Zane Bitter}}
\date{24 October 2013}

\usepackage{pdfpages}

\begin{document}
\maketitle

\marginnote{\tableofcontents \vspace{2em}}

\begin{abstract}
An introduction to the OpenStack Orchestration project, Heat, and an explanation of how orchestration can help simplify the deployment and management of your cloud application by allowing you to represent infrastructure as code. Some of the major new features in Havana are also covered, along with a preview of development plans for Icehouse.
\end{abstract}

\section{What is Heat?}

Heat is the project providing Orchestration for OpenStack. It allows cloud infrastructure to be represented in a declarative form (`template'). The service itself parses these templates and calls the various OpenStack APIs to create or modify the deployment, while maintaining correct dependencies between them. The collection of resources associated with a template is known as a stack. One of the resource types available is a stack, so templates can be composed into hierarchical structures.

Heat is based on the AWS CloudFormation service. CloudFormation templates are represented as \textsc{Json} documents. Heat has added a \textsc{Yaml} serialisation for easier consumption by humans, as well as OpenStack native resource types in addition to AWS compatibility resource types. (Heat also has its own OpenStack-native \textsc{Rest} API in addition to a CloudFormation-compatible Query API.)

CloudFormation templates are widely held to be inferior to \textsc{Tosca} service templates for deploying complex applications, because CloudFormation is designed around orchestrating infrastructure resources, not the software configuration of applications.


\section{Heat Templates}

\begin{marginfigure}
\begin{verbatim}
Parameters:
  ssh_key_name:
    Type: String
    Description: ssh keypair name
  image_name:
    Type: String
    Description: The image to boot

Resources:
  my_server:
    Type: OS::Nova::Server
    Properties:
      flavor: m1.small
      key_name: {"Ref": "ssh_key"}
      block_device_mapping:
        device_name: vda
        volume_id: {"Ref": "my_vol"}

  my_vol:
    Type: OS::Cinder::Volume
    Properties:
      size: 20
      image: {"Ref": "image_name"}

Outputs:
  server_ip:
    Description: The server IP
    Value: {"Fn::GetAtt":
                ["my_server",
                 "first_address"]}
\end{verbatim}

\caption{A simple example template that creates a Nova server and an attached Cinder volume. The user can specify the image to boot from and the name of the ssh public key to install on the server, and the IP address of the server is available as an output.}
\label{fig:example-template}
\end{marginfigure}

Heat templates usually take the form of simple \textsc{Yaml} documents. We chose this over \textsc{Json}, which CloudFormation uses, because it is much easier for humans to read and write---and a diff between two versions of a \textsc{Yaml} template is usually trivial to interpret. \textsc{Yaml} is a strict superset of \textsc{Json} though, so \textsc{Json} is still fully supported and templates can be converted between the two formats with no loss of fidelity.

Other than the serialisation format, Heat (for now) adheres closely to the CloudFormation template model. (An example template is shown in Figure~\ref{fig:example-template} at right.) A template has four key elements:

\begin{itemize}
\item An optional \texttt{Parameters} section, which allows user-definable inputs to be specified.
\item An optional \texttt{Mappings} section, which allows key/value lookup of predefined constants.
\item A mandatory \texttt{Resources} section, which describes the resources and relationships which define the application, configuration and infrastructure.
\item An optional \texttt{Outputs} section, which describes the output values to be returned to the user.
\end{itemize}

All resources have a common interface:

\begin{itemize}
\item A number of optional or mandatory \texttt{Properties} which specify inputs that affect how the resource is configured.
\item A number of output attributes, which may be referenced elsewhere in the template using the \texttt{Fn::GetAtt} function.
\end{itemize}

One of the resource types available is a Heat stack, so multiple templates can be composed into hierarchical structures.

A resource may reference any other resource or its attributes, which causes Heat to infer an ordering dependency between them. The template may also specify explicit dependencies where necessary. Lifecycle operations\footnote{That is to say, operations such as creating, deleting or updating a stack.} occur in parallel to the maximum extent possible given the dependencies between resources.

A number of functions are available which allow some limited manipulation of data within the template, for example to combine parameter data with static data in the template.


\section{Autoscaling}


\section{Providers and Environments}



\section{Matahari Roadmap}

A detailed list of what tasks are currently being worked on can be found on the project \href{https://github.com/matahari/matahari/wiki/Backlog}{backlog}.  There are some additional things worth mentioning, though.

The Matahari team plans to get more involved in the development of QMF itself.  We have written some code that helps make writing QMF (and D-Bus) agents easier, and we would like to get some of those enhancements merged back into QMF.  Some of the other things we would like to do in QMF are to help drive adoption.  This includes splitting QMF out from the Qpid project, a re-work in C (it is currently C++), and potentially supporting other underlying AMQP implementations to be able to interoperate with AMQP infrastructure that doesn't use the same version of AMQP as Qpid.

In addition to working on the infrastructure, we will also be continuing to write agents.  Matahari will grow as a collection of general-purpose cross-platform agents and we will be working with with other projects to expose their APIs via the Matahari infrastructure.

\subsection{More Information}

Here is a list of other places you can go to get more information and participate in the project:

\begin{itemize}
\item The project home page is \url{http://matahariproject.org/}.
\item You can find the code on \href{https://github.com/matahari/matahari}{GitHub}.
\item You can follow patches and project-related discussion on the \href{https://fedorahosted.org/mailman/listinfo/matahari}{Matahari mailing list}. 
\item You can join the \texttt{\#matahari} IRC channel on OFTC.
\end{itemize}


\begin{marginfigure}
\section{More Information}
\begin{itemize}
\item \href{https://wiki.openstack.org/Heat}{`Heat' page on the OpenStack Wiki}
\item \href{http://docs.openstack.org/developer/heat/#getting-started}{Getting Started guides}
\item \href{http://openstack.redhat.com/Docs}{RDO Documentation}
\item \href{https://ask.openstack.org/}{Ask OpenStack}
\item \href{https://twitter.com/zerobanana}{@zerobanana}
\item \href{http://www.zerobanana.com/tags/OpenStack}{zerobanana.com}
\end{itemize}
\end{marginfigure}

\end{document}
